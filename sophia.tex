%%%%%%%%%%%%%%%%%%%
% Sophia Template %
%%%%%%%%%%%%%%%%%%%

%%%%%%%%%%%%
% Sophia is a very basic LaTeX template that is completely free to use, modify and distribute.
% It is intended for works in the humanities and social sciences
% It has a simple yet classy look and some basic features which should be enough for most papers and theses.
% This is a template for beginners by a beginner, so don't expect anything too fancy.
% For suggestions and issues, head over to https://gitlab.com/teunphil/sophia-template
%%%%%%%%%%%%


%%%%%%%%%%%%
% Preamble %
%%%%%%%%%%%%

%%% Document class %%%
% Any arguments given here will be passed on to all packages that support it.
% This is useful if multiple packages take the same type of argument.
% For example, both Babel and BibLaTeX-Chicago take languages as arguments.
% Thus, the language is specified once here so that it can be passed on to both packages.
% Options: USenglish, UKenglish, dutch
% Other useful options include:
%	twoside – for double-sided printing
\documentclass[UKenglish]{article}
%%% Author name %%%
\author{Sophia Q. Smart}

%%% Title %%%
\title{\textbf{Title}\\Subtitle}

%%% Date %%%
% For no date, write \date{}
\date{\today}

%%% Font %%%
% Set font encoding to T1
% Comment to restore to LaTeX default (OT1)
% If you have no idea what this is, it's best to leave it
\usepackage[T1]{fontenc}

% I use TeX Gyre Termes because I think it's pretty
% Comment to restore to LaTeX default (Computer Modern)
\usepackage{tgtermes}

%%% Paper size %%%
% Options: a4paper, letterpaper (LaTeX default)
\usepackage[a4paper]{geometry}

%%% Spacing %%%
\usepackage{setspace}
% Options: singlespacing (default), onehalfspacing, doublespacing
\onehalfspacing

%%% Microtype %%%
% Microtype makes a bunch of small typesetting changes to make your document look better
% Comment to speed up compiling, or to see what the differences are
\usepackage{microtype}

%%% Babel %%%
% Sets document language based on the argument passed to the \documentclass variable
\usepackage{babel}

%%% BibLaTeX %%%
% Enables BibLaTeX-Chicago in the author-date format, with Biber as backend, and suppressing all doi's and isbn's
% Biber needs to be installed on your machine for this to work!
% Change 'authordate' to 'notes' to use footnotes instead
\usepackage[
authordate,
backend=biber,
doi=false,
isbn=false]
{biblatex-chicago}

% Sets the bibliography source file
% Takes both absolute and relative paths, so all of the following are valid:
%	/home/name/Documents/article/sources.bib
%	sources.bib
%	../sources/sources.bib
\addbibresource{filename.bib}

% Enables csquotes, which is recommended for BibLaTeX-Chicago
\usepackage[autostyle=true]{csquotes}

% Set bibliography font to small (or even smaller)
\renewcommand{\bibfont}{\normalfont\small}
%\renewcommand{\bibfont}{\normalfont\footnotesize}

%%% References %%%
% Enables Hyperref, turning references into hyperlinks
\usepackage{hyperref}
% Set custom colors for urls, links and citations
\hypersetup{
  colorlinks   = true, %Colours links instead of ugly boxes:
  urlcolor     = blue, %Colour for external hyperlinks
  linkcolor    = red, %Colour of internal links
  citecolor   = blue %Colour of citations
}

% The 'caption' package enables the [hypcap=true] option (enabled by default), which lets hyperref refer to the figure itself instead of the caption
\usepackage{caption}

% Enables Cleveref, which partly automates internal references
% Instead of 'see paragraph \ref{sec:analysis}', write 'see \cref{sec:analysis}'
% Cleveref will automatically append the type of thing you're referencing (e.g. paragraph, figure, appendix, etc.) and do so in the right language
\usepackage{cleveref}

% Enables line breaks in URLs and file paths
\usepackage{url}

% Enables comment blocks
\usepackage{comment}


%%%%%%%%
% Body %
%%%%%%%%

\begin{document}

\maketitle

\begin{abstract}

\end{abstract}

\textit{\textbf{\small Keywords --- }\small first, second, third, etc.}

\section{A section}
This is where the good stuff happens.


\subsection{A subsection}
Also very interesting yes.
Oh, and here is \href{https://youtu.be/dQw4w9WgXcQ}{a link}!


\subsubsection{A subsubsection}
Subsubsections exist too, apparently.

\paragraph{A paragraph}
I'm going deeper underground.

\subparagraph{A subparagraph}
There's too much panic in this town!

% To add a sub-header to a section header, simply use \\, change the size and keep typing
% In order to make header and sub-header appear properly in the ToC, use square brackets and type the header you want to have in the ToC
\section[Section two: the sectioning]{Section two \\ \large{The sectioning}}
This is a section with a subtitle for extra \emph{emphasis}.
\textbf{Bold} stuff!
As for other formatting, ``A short quote can go between quotation marks'', said the author of this template.

\begin{quote}
	A longer quote, such as the present one, requires a different approach.
	Due to the length of this text, it would be preferred to visually separate it from the rest of the text.
	For that purpose, a block quote is required.
	And, it should be said, it does the job marvellously!
\end{quote}

Citing is usually done with \verb|\autocite{tag}| or, to cite in-line, \verb|\textcite{tag}|.
Unfortunately, these cannot be demonstrated here as they need a .bib file to work.

% The printbibliography command will only work once at least one source has been cited in the document
% Otherwise, it will throw a warning
\printbibliography

\end{document}
