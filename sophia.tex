%%%%%%%%%%%%%%%%%%%
% Sophia Template %
%%%%%%%%%%%%%%%%%%%

%%%%%%%%%%
% Sophia is a very basic LaTeX template that anyone is completely free to use, modify and distribute.
% It is mostly intended for philosophy works, but it might be suitable for other disciplines in the humanities and social sciences as well.
% It has a simple yet classy look and some basic features which should be enough for most papers and theses.
% This is a template for beginners by a beginner, so don't expect anything too fancy.
% For suggestions and issues, head over to https://gitlab.com/teunphil/sophia-template
%%%%%%%%%%


%%% Document class %%%
% Any arguments given here will be passed on to all packages that support it.
% This is useful if multiple packages take the same type of argument.
% For example, both Babel and BibLaTeX-Chicago take languages as arguments.
% Thus, the language is specified once here so that it can be passed on to both packages.
% Options: USenglish, UKenglish, dutch
% Other useful options include:
%	twocolumn – for two columns of text per page
%	twoside – for double-sided printing
%	titlepage – to put the title centered on a separate page
\documentclass[UKenglish, titlepage]{article}
%\documentclass[UKenglish, twocolumn]{article}
%%% Author name %%%
\author{Teun van Son}

%%% Title %%%
\title{\textbf{Title}\\Subtitle}

%%% Date %%%
% For no date, write \date{}
\date{\today}

%%% Font %%%
% Set font encoding to T1
% Comment to restore to LaTeX default (OT1)
% If you have no idea what this is, it's best to leave it
\usepackage[T1]{fontenc}

% I use TeX Gyre Termes because I think it's pretty
% Comment to restore to LaTeX default (Computer Modern)
\usepackage{tgtermes}

%%% Paper size %%%
% Options: a4paper, letterpaper (LaTeX default)
\usepackage[a4paper]{geometry}

%%% Spacing %%%
\usepackage{setspace}
% Options: singlespacing, onehalfspacing, doublespacing
\singlespacing

%%% Microtype %%%
% Microtype makes a bunch of small typesetting changes to make your document look better
% Commented by default to speed up compiling
% Uncomment to compile the final version of your document
%\usepackage{microtype}

%%% Booktabs %%%
% Makes tables simple and pretty
% Enabled by default, comment the line below if your document has no tables
\usepackage{booktabs}

%%% Babel %%%
% Sets document language based on the argument passed to the \documentclass variable
\usepackage{babel}

%%% BibLaTeX %%%
% Enables BibLaTeX-Chicago in the author-date format, with Biber as backend, and suppressing all doi's and isbn's
% Biber needs to be installed on your machine for this to work!
% Change 'authordate' to 'notes' to use footnotes instead
\usepackage[
authordate,
backend=biber,
doi=false,
isbn=false]
{biblatex-chicago}
% Sets the bibliography source file
% Takes both absolute and relative paths, so all of the following are valid:
%	/home/name/Documents/article/sources.bib
%	sources.bib
%	../sources/sources.bib
\addbibresource{filename.bib}

% Enables csquotes, which is recommended for BibLaTeX-Chicago
\usepackage[autostyle=true]{csquotes}

% Set bibliography font to small
\renewcommand{\bibfont}{\normalfont\small}
%\renewcommand{\bibfont}{\normalfont\footnotesize}

%%% References %%%
% Enables Hyperref, turning references into hyperlinks
\usepackage{hyperref}
% Set custom colors for urls, links and citations
\hypersetup{
  colorlinks   = true, %Colours links instead of ugly boxes
  urlcolor     = blue, %Colour for external hyperlinks
  linkcolor    = red, %Colour of internal links
  citecolor   = blue %Colour of citations
}

% Enables the [hypcap=true] option (enabled by default) that lets hyperref refer to the figure itself instead of the caption
\usepackage{caption}

% Enables Cleveref, which partly automates internal references
% Instead of 'see paragraph \ref{sec:analysis}', write 'see \cref{sec:analysis}'
% Cleveref will automatically append the type of thing you're referencing (e.g. paragraph, figure, appendix, etc.)
\usepackage{cleveref}

%%% Graphics %%%
% Enables insertion of images
% Disabled by default, uncomment the two lines below if your document needs images
%\usepackage{graphicx}
%\graphicspath{{../Afbeeldingen}}


%%% Document body %%%
\begin{document}

\maketitle

% Puts the title on its own page; recommended for bigger documents such as theses
% Comment the line below to let the text continue below the title
%\clearpage

%%% Table of Contents %%%
%{ 
%\hypersetup{linkcolor=black} %To make sure that the links in the table of contents aren't red
%\tableofcontents
%}
%\pagebreak

\section{A section}
This is where the good stuff happens.

\subsection{A subsection}
Also very interesting yes.
Oh, and here is \href{https://youtu.be/dQw4w9WgXcQ}{a link}!


%%% Tables %%%
\section{Table}

% This is a simple example table; feel free to modify it to your needs.
% Uncomment the four commented lines to put this table in a floating figure.
\begingroup % A group is created to quarantine the distance parameters set below
\setlength{\tabcolsep}{9pt} % Default value: 6pt
\renewcommand{\arraystretch}{1.5} % Default value: 1
%\begin{figure}
\begin{tabular}{p{0.45\textwidth} p{0.45\textwidth}} \toprule
	\textbf{Header 1} & \textbf{Header 2} \\
	\midrule
	Some text on the left. & Some more text on the right. \\
	This is a somewhat longer though not unreasonably long piece of text. & This is short. \\
	Etc. etc. & You probably get the point by now. \\	
		\bottomrule
\end{tabular}
%\caption{A very simple example table.}
%\label{fig:table}
%\end{figure}
\endgroup

%%% Images %%%

% By default, figures appear in the most space-saving spot.
% You can change this behaviour by specifying an option between brackets after \begin{figure}:
%	'h' puts them exactly where you assign them.
%	't' puts them at the top.
%	'b' puts them at the bottom.
%	'p' puts them on their own page.
% Uncomment the next section to include an image:
%\begin{figure}[h]
%	\begin{center}
%		\includegraphics[width=0.5\textwidth]{image.png}
%		\caption{A very beautiful image.}
%		\label{fig:extinctions}
%	\end{center}
%\end{figure}

\printbibliography


%%% Appendix %%%
% Adds an appendix to the document

%\pagebreak

% Enables separate numbering for the appendix
% Options: arabic, roman, Roman, alph, Alph
%\pagenumbering{roman}

% Creates a separate section of references for the appendix bibliography
%\begin{refsection}

%\appendix

%\section{First appendix}

% Prints the appendix bibliography with a smaller heading
%\printbibliography[heading=subbibliography]

%\end{refsection}


\end{document}
