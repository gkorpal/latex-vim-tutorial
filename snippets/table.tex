%%%%%%%%%%%%
% Preamble %
%%%%%%%%%%%%
% Making tables in LaTeX can be done without any extra packages
% However, booktabs makes tables simple and pretty
\usepackage{booktabs}

%%%%%%%%%%%%%
% Body text %
%%%%%%%%%%%%%
% This is a simple example table; feel free to modify it to your needs.
% By default, this table is put in a floating figure.
% (For more info on figures, see the `image' snippet)
% To put this table in-line instead, comment the lines \begin{figure}, \caption, \label and \end{figure}
\begingroup % A group is created to quarantine the distance parameters set below
% The two lines below increase the distance between columns (\tabcolsep) and between rows (\arraystretch).
\setlength{\tabcolsep}{9pt} % Default LaTeX value: 6pt – Default Sophia value: 9pt
\renewcommand{\arraystretch}{1.5} % Default LaTeX value: 1 – Default Sophia value: 1.5
\begin{figure}
% This table has two paragraphs that wrap their text automatically (hence the p's).
% Besides 'p', other options include c (center), l (left align) and r (right align).
\begin{tabular}{p{0.45\textwidth} p{0.45\textwidth}} \toprule
	\textbf{Header 1} & \textbf{Header 2} \\
	\midrule
	Some text on the left. & Some more text on the right. \\
	This is a somewhat longer though not unreasonably long piece of text. & This is short. \\
	Etc. etc. & You probably get the point by now. \\
		\bottomrule
\end{tabular}
\caption{A very simple example table.}
\label{fig:table}
\end{figure}
\endgroup
